\item[Classroom Behavior:] 
Students and faculty are responsible for maintaining an appropriate learning environment in all instructional settings, whether in person, remote, or online. 
Failure to adhere to such behavioral standards may be subject to discipline. 
Professional courtesy and sensitivity are especially important with respect to individuals and topics dealing with race, color, national origin, sex, pregnancy, age, disability, creed, religion, sexual orientation, gender identity, gender expression, veteran status, political affiliation, or political philosophy.
  
For more information, see the \href{http://www.colorado.edu/policies/student-classroom-and-course-related-behavior}{classroom behavior policy}, the \href{https://www.colorado.edu/sccr/student-conduct}{Student Code of Conduct}, and the \href{https://www.colorado.edu/oiec/}{Office of Institutional Equity and Compliance}. 

\item[Requirements for Infectious Diseases:]
Members of the CU Boulder community and visitors to campus must follow university, department, and building health and safety requirements and all public health orders to reduce the risk of spreading infectious diseases.
 
The CU Boulder campus is currently mask optional. 
However, if masks are again required in classrooms, students who fail to adhere to masking requirements will be asked to leave class. 
Students who do not leave class when asked or who refuse to comply with these requirements will be referred to Student Conduct \& Conflict Resolution. 
Students who require accommodation because a disability prevents them from fulfilling safety measures related to infectious disease will be asked to follow the steps in the ``Accommodation for Disabilities'' statement on this syllabus.

For those who feel ill and think you might have COVID-19 or if you have tested positive for COVID-19, please stay home and follow the \href{https://www.colorado.edu/healthcenter/coronavirus-updates/symptoms-and-what-do-if-you-feel-sick}{further guidance of the Public Health Office}. 
For those who have been in close contact with someone who has COVID-19 but do not have any symptoms and have not tested positive for COVID-19, you do not need to stay home. 

\item[Accommodation for Disabilities, Temporary Medical Conditions, and Medical Isolation:] 
\href{https://www.colorado.edu/disabilityservices/}{Disability Services} determines accommodations based on documented disabilities in the academic environment. 
If you qualify for accommodations because of a disability, submit your accommodation letter from Disability Services to your faculty member in a timely manner so your needs can be addressed. 
Contact Disability Services at 303-492-8671 or \href{mailto:dsinfo@colorado.edu}{dsinfo@colorado.edu} for further assistance.
  
If you have a temporary medical condition or required medical isolation for which you require accommodation,  please notify the instructor as soon as possible so that appropriate accommodations can be made.
If you are sick or require isolation please notify the instructor of your absence from in-person activities and continue in a completely remote mode, as you are able, until you are allowed or able to return to campus.
Please note that for health privacy reasons you are not required to disclose to the instructor the nature of your illness or condition, however you are welcome to share information you feel necessary to protect the health and safety of others within the course.
Also see \href{http://www.colorado.edu/disabilityservices/students/temporary-medical-conditions}{Temporary Medical Conditions} on the Disability Services website.
 
\item[Preferred Student Names and Pronouns:]
CU Boulder recognizes that students' legal information doesn't always align with how they identify. 
Students may update their preferred names and pronouns via the student portal; those preferred names and pronouns are listed on instructors' class rosters. 
In the absence of such updates, the name that appears on the class roster is the student's legal name.

\item[Honor Code:] 
All students enrolled in a University of Colorado Boulder course are responsible for knowing and adhering to the  \href{https://www.colorado.edu/sccr/honor-code}{Honor Code}. 
Violations of the Honor Code may include but are not limited to: plagiarism (including use of paper writing services or technology [such as essay bots]), cheating, fabrication, lying, bribery, threat, unauthorized access to academic materials, clicker fraud, submitting the same or similar work in more than one course without permission from all course instructors involved, and aiding academic dishonesty. 

All incidents of academic misconduct will be reported to Student Conduct \& Conflict Resolution: \href{mailto:honor@colorado.edu}{honor@colorado.edu}, 303-492-5550. 
Students found responsible for violating the \href{https://www.colorado.edu/sccr/honor-code}{Honor Code} will be assigned resolution outcomes from the Student Conduct \& Conflict Resolution as well as be subject to academic sanctions from the faculty member. 
Visit \href{https://www.colorado.edu/sccr/honor-code}{Honor Code} for more information on the academic integrity policy.

\item[Sexual Misconduct, Discrimination, Harassment and/or Related Retaliation:] 
CU Boulder is committed to fostering an inclusive and welcoming learning, working, and living environment.
University policy prohibits sexual misconduct (harassment, exploitation, and assault), intimate partner violence (dating or domestic violence), stalking, protected-class discrimination and harassment, and related retaliation by or against members of our community on and off campus.

Visit \href{http://www.colorado.edu/institutionalequity/}{OIEC} for or more information about university policies, \href{https://www.colorado.edu/oiec/reporting-resolutions/making-report}{reporting options}, and support resources. 
If you believe you may have been subjected to misconduct, \href{mailto:cureport@colorado.edu}{email OIEC} or call 303-492-2127.

Faculty and graduate instructors are required to inform OIEC when they learn of any issues related to these policies regardless of when or where they occurred. 
This ensures that individuals impacted receive information about their rights, support resources, and resolution options. 
Visit the \href{https://www.colorado.edu/dontignoreit/}{Don’t Ignore It} page to learn more about reporting and support options.

\item[Religious Holidays:] 
Campus policy regarding religious observances requires that faculty make every effort to deal reasonably and fairly with all students who, because of religious obligations, have conflicts with scheduled exams, assignments or required attendance. 
In this class, you must let the instructor know of any such conflicts within the first two weeks of the semester so that they can work with you to make reasonable arrangements.

See the \href{http://www.colorado.edu/policies/observance-religious-holidays-and-absences-classes-andor-exams}{campus policy regarding religious observances} for full details.