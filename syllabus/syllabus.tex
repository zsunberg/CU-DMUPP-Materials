\documentclass[9pt]{article}

\usepackage{fullpage}
\usepackage{hyperref}
\usepackage{enumitem}
\usepackage{multicol}
\usepackage[normalem]{ulem}

\addtolength{\topmargin}{-.25in}
\addtolength{\textheight}{0.5in}
\setlength{\parindent}{0pt}
\setlength{\multicolsep}{2pt}

\title{ASEN 6519: Advanced Survey of Sequential Decision Making}
\author{Zachary Sunberg}
\date{Fall 2025}

\begin{document}

\maketitle

\section*{Prerequisites}

ASEN 5264 Decision Making under Uncertainty, research experience in robotics or artificial intelligence, or permission from the instructor.

\section*{Learning Objectives}

\begin{enumerate}[nosep]
    \item Obtain a broad understanding of models and algorithms for sequential decision making
    \item Understand theoretical tools for analyzing sequential decision making
    \item Communicate about recent research in sequential decision making
    \item Become familiar with and investigate open research questions in sequential decision making
\end{enumerate}

\section*{List of Topics}

\begin{enumerate}[nosep]
    \item Survey of advanced reinforcement learning algorithms
    \item Theoretical tools for characterizing sequential decision-making algorithms
    \item Important properties of select algorithms
    \item Difficulties inherent in multi-agent interaction and solution approaches for them
    \item Guest Lectures by researchers in the field
\end{enumerate}

\section*{Assignments and Grading}

\textbf{40\% Presentations.}
Each student will read and prepare presentations for three papers that will be assigned after the first week. For each paper there will be two requirements:
\begin{itemize}[nosep]
    \item Draft presentation: Must be uploaded several days before the date of the presentation - see Ed for details.
    \item In-class presentation and discussion: You will deliver a 20-25 minute presentation summarizing the paper and lead a 10-15 minute discussion.
\end{itemize}

\textbf{10\% Understanding Assessments.}
There will be 3-4 short assessments (quizzes or exercises) to assess understanding of the papers we read in the course.

\textbf{10\% Blog Post and Oral Exam.}
Students will prepare a blog post on one of the topics in the course and will take a 1-on-1 oral exam on the content of the blog post with the professor.

\textbf{35\% Final Project.}
A final project chosen by the student that ideally connects to their research. Deliverables will be a 15 minute presentation and a 4-8 page report. The project may be completed in teams of up to 2.

\textbf{5\% Participation.}
Though the workload for this course is smaller than the average graduate course at CU, \textbf{excellence} and \textbf{engagement} are highly valued. Students are expected to attend all lectures except for approved reasons such as travel, illness, or religious observances. You will also be expected to ask questions and participate in discussions during the semester.

\section*{Learning Technology}

\textbf{Ed} (\href{https://edstem.org/us/courses/83529}{edstem.org/us/courses/83529}) will be the main hub for the course. All other materials and instructions will be posted there.

\subsection*{Use of AI in This Ccourse}

You are \textbf{expected} to use artificial intelligence such as large language models to accelerate your learning in this class. However, \textbf{you are responsible for the accuracy} of any content that you produce; you must \textbf{understand} any content produced by artificial intelligence; and you must \textbf{attribute} or note the use of generative AI when you use it. There may be exceptions to this policy for individual assignments. These will be noted on the assignment.

\section*{Instructor Contact}

Professor Zachary Sunberg\\
AERO 263 \href{mailto://zachary.sunberg@colorado.edu}{zachary.sunberg@colorado.edu}\\
\textbf{Office Hours}
By request: please create a calendar invitation for a time not marked busy on my calendar at \url{ https://zachary.sunberg.net/contact#calendar} and indicate whether you would like to meet in my office or provide a videoconference link. \\

\section*{Meetings}

Tuesdays and Thursdays 1-2:15 pm\\
AERO 114

\section*{University Policies}

\subsection*{Honor Code}

All students enrolled in a University of Colorado Boulder course are responsible for knowing and adhering to the \href{https://www.colorado.edu/sccr/students/honor-code-and-student-code-conduct}{Honor Code}. Violations of the Honor Code may include but are not limited to: plagiarism (including use of paper writing services or technology [such as essay bots]), cheating, fabrication, lying, bribery, threat, unauthorized access to academic materials, clicker fraud, submitting the same or similar work in more than one course without permission from all course instructors involved, and aiding academic dishonesty. Understanding the course's syllabus is a vital part of adhering to the Honor Code.

All incidents of academic misconduct will be reported to Student Conduct \& Conflict Resolution: \href{mailto:StudentConduct@colorado.edu}{StudentConduct@colorado.edu}. Students found responsible for violating the Honor Code will be assigned resolution outcomes from Student Conduct \& Conflict Resolution and will be subject to academic sanctions from the faculty member. Visit \href{https://www.colorado.edu/sccr/students/honor-code-and-student-code-conduct}{Honor Code} for more information on the academic integrity policy.

\subsection*{Accommodation for Disabilities, Temporary Medical Conditions, and Medical Isolation}

If you qualify for accommodations because of a disability, please submit your accommodation letter from Disability Services to your faculty member in a timely manner so that your needs can be addressed. Disability Services determines accommodations based on documented disabilities in the academic environment. Information on requesting accommodations is located on the \href{https://www.colorado.edu/disabilityservices/}{Disability Services website}. Contact Disability Services at 303-492-8671 or DSinfo@colorado.edu for further assistance. If you have a temporary medical condition, see \href{https://www.colorado.edu/disabilityservices/students/temporary-medical-conditions}{Temporary Medical Conditions} on the Disability Services website.

If you have a temporary illness, injury or required medical isolation for which you require adjustment, please contact the instructor via email.

\subsection*{Accommodation for Religious Obligations}

Campus policy requires faculty to provide reasonable accommodations for students who, because of religious obligations, have conflicts with scheduled exams, assignments, or required attendance. Please communicate the need for a religious accommodation in a timely manner. See the \href{https://www.colorado.edu/compliance/policies/observance-religious-holidays-absences-classes-or-exams}{campus policy regarding religious observances} for full details. Please contact the instructor via email for 

\subsection*{Preferred Student Names and Pronouns}

CU Boulder recognizes that students' legal information does not always align with how they identify. If you wish to have your preferred name (rather than your legal name) and/or your preferred pronouns appear on your instructors' class rosters and in Canvas, visit the \href{https://www.colorado.edu/registrar/students/records/info/preferred}{Registrar's website} for instructions on how to change your personal information in university systems.

\subsection*{Classroom Behavior}

Students and faculty are responsible for maintaining an appropriate learning environment in all instructional settings, whether in person, remote, or online. Failure to adhere to such behavioral standards may be subject to discipline. Professional courtesy and sensitivity are especially important with respect to individuals and topics dealing with race, color, national origin, sex, pregnancy, age, disability, creed, religion, sexual orientation, gender identity, gender expression, veteran status, marital status, political affiliation, or political philosophy.

\subsubsection*{Additional classroom behavior information}

\begin{itemize}
    \item \href{https://www.colorado.edu/compliance/policies/student-classroom-course-related-behavior}{Student Classroom and Course-Related Behavior Policy}.
    \item \href{https://www.colorado.edu/sccr/students/honor-code-and-student-code-conduct}{Student Code of Conduct}.
    \item \href{https://www.colorado.edu/oiec/}{Office of Institutional Equity and Compliance}.
    \item \href{https://www.colorado.edu/sccr/students/honor-code-and-student-code-conduct}{Student Code of Conduct}.
    \item \href{https://www.colorado.edu/oiec/}{Office of Institutional Equity and Compliance}.
\end{itemize}

\subsection*{Sexual Misconduct, Discrimination, Harassment and/or Related Retaliation}

CU Boulder is committed to fostering an inclusive and welcoming learning, working, and living environment. University policy prohibits \href{https://www.colorado.edu/oiec/policies/protected-class-nondiscrimination-policy/protected-class-definitions}{protected-class} discrimination and harassment, sexual misconduct (harassment, exploitation, and assault), intimate partner abuse (dating or domestic violence), stalking, and related retaliation by or against members of our community on- and off-campus. The Office of Institutional Equity and Compliance (OIEC) addresses these concerns, and individuals who have been subjected to misconduct can contact OIEC at 303-492-2127 or email OIEC@colorado.edu. Information about university policies, \href{https://www.colorado.edu/oiec/reporting-resolutions/making-report}{reporting options}, and \href{https://www.colorado.edu/oiec/support-resources}{OIEC support resources} including confidential services can be found on the \href{https://www.colorado.edu/oiec/}{OIEC website}.

Please know that faculty and graduate instructors are required to inform OIEC when they are made aware of incidents related to these concerns regardless of when or where something occurred. This is to ensure the person impacted receives outreach from OIEC about resolution options and support resources. To learn more about reporting and support a variety of concerns, visit the \href{https://www.colorado.edu/dontignoreit/}{Don't Ignore It page}.

\subsection*{Mental Health and Wellness}

The University of Colorado Boulder is committed to the well-being of all students. If you are struggling with personal stressors, mental health or substance use concerns that are impacting academic or daily life, please contact \href{https://www.colorado.edu/counseling/}{Counseling and Psychiatric Services (CAPS)}, located in C4C, or call (303) 492-2277, 24/7.




\end{document}
