\documentclass[9pt]{article}

\usepackage{fullpage}
\usepackage{hyperref}
\usepackage{enumitem}
\usepackage{multicol}
\usepackage[normalem]{ulem}

\addtolength{\topmargin}{-.25in}
\addtolength{\textheight}{0.5in}
\setlength{\parindent}{0pt}
\setlength{\multicolsep}{2pt}

\title{ASEN 6519: Advanced Survey of Sequential Decision Making}
\author{Zachary Sunberg}
\date{Fall 2025}

\begin{document}

\maketitle

\section*{Prerequisites}

ASEN 5264 Decision Making under Uncertainty, research experience in robotics or artificial intelligence, or permission from the instructor.

\section*{Learning Objectives}

\begin{enumerate}[nosep]
    \item Obtain a broad understanding of models and algorithms for sequential decision making
    \item Understand theoretical tools for analyzing sequential decision making
    \item Communicate about recent research in sequential decision making
    \item Become familiar with and investigate open research questions in sequential decision making
\end{enumerate}

\section*{List of Topics}

\begin{enumerate}[nosep]
    \item Survey of advanced reinforcement learning algorithms
    \item Theoretical tools for characterizing sequential decision-making algorithms
    \item Important properties of select algorithms
    \item Difficulties inherent in multi-agent interaction and solution approaches for them
    \item Guest Lectures by researchers in the field
\end{enumerate}

\section*{Assignments and Grading}

\textbf{40\% Presentations.}
Each student will read and prepare presentations for three papers that will be assigned after the first week. For each paper there will be two requirements:
\begin{itemize}[nosep]
    \item Draft presentation: Must be uploaded several days before the date of the presentation - see Ed for details.
    \item In-class presentation and discussion: You will deliver a 20-25 minute presentation summarizing the paper and lead a 10-15 minute discussion.
\end{itemize}

\textbf{10\% Understanding Assessments.}
There will be 3-4 short assessments (quizzes or exercises) to assess understanding of the papers we read in the course.

\textbf{10\% Blog Post and Oral Exam.}
Students will prepare a blog post on one of the topics in the course and will take a 1-on-1 oral exam on the content of the blog post with the professor.

\textbf{35\% Final Project.}
A final project chosen by the student that ideally connects to their research. Deliverables will be a 15 minute presentation and a 4-8 page report. The project may be completed in teams of up to 2.

\textbf{5\% Participation.}
Though the workload for this course is smaller than the average graduate course at CU, \textbf{excellence} and \textbf{engagement} are highly valued. Students are expected to attend all lectures except for approved reasons such as travel, illness, or religious observances. You will also be expected to ask questions and participate in discussions during the semester.

\section*{Learning Technology}

\textbf{Ed} (\href{https://edstem.org/us/courses/83529}{edstem.org/us/courses/83529}) will be the main hub for the course. All other materials and instructions will be posted there.

\subsection*{Use of AI in This Ccourse}

You are \textbf{expected} to use artificial intelligence such as large language models to accelerate your learning in this class. However, \textbf{you are responsible for the accuracy} of any content that you produce; you must \textbf{understand} any content produced by artificial intelligence; and you must \textbf{attribute} or note the use of generative AI when you use it. There may be exceptions to this policy for individual assignments. These will be noted on the assignment.

\section*{Instructor Contact}

Professor Zachary Sunberg\\
AERO 263 \href{mailto://zachary.sunberg@colorado.edu}{zachary.sunberg@colorado.edu}\\
\textbf{Office Hours}
By request: please create a calendar invitation for a time not marked busy on my calendar at \url{ https://zachary.sunberg.net/contact#calendar} and indicate whether you would like to meet in my office or provide a videoconference link. \\

\section*{Meetings}

Tuesdays and Thursdays 1-2:15 pm\\
AERO 114

\section*{University Policies}

\input{2025-required-syllabus-statements}

\end{document}
